\documentclass[a4paper,12pt]{article}

\usepackage[utf8]{inputenc}
\usepackage[vietnamese]{babel} % Hỗ trợ tiếng Việt
\usepackage{graphicx}
\usepackage{geometry}
\usepackage{listings}
\usepackage{xcolor}
\usepackage{float}
\usepackage{hyperref}
\usepackage{tikz}
\usetikzlibrary{shapes, arrows, positioning, calc, shadows}

\geometry{
  top=2cm,
  bottom=2cm,
  left=2.5cm,
  right=2.5cm
}

\definecolor{codegreen}{rgb}{0,0.6,0}
\definecolor{codegray}{rgb}{0.5,0.5,0.5}
\definecolor{codepurple}{rgb}{0.58,0,0.82}
\definecolor{backcolour}{rgb}{0.95,0.95,0.92}

\lstdefinestyle{mystyle}{
    backgroundcolor=\color{backcolour},   
    commentstyle=\color{codegreen},
    keywordstyle=\color{magenta},
    numberstyle=\tiny\color{codegray},
    stringstyle=\color{codepurple},
    basicstyle=\ttfamily\footnotesize,
    breakatwhitespace=false,         
    breaklines=true,                 
    captionpos=b,                    
    keepspaces=true,                 
    numbers=left,                    
    numbersep=5pt,                  
    showspaces=false,                
    showstringspaces=false,
    showtabs=false,                  
    tabsize=2
}

\lstset{style=mystyle}

\title{\textbf{Practical Work 1: TCP File Transfer}}
\author{Nhóm thực hiện} % Bạn điền tên nhóm vào đây
\date{\today}

\begin{document}

\begin{titlepage}
    \begin{center}
        \vspace*{1cm}
        
        \Huge
        \textbf{University of Science and Technology of Hanoi}
        
        \vspace{0.5cm}
        \LARGE
        Department of Information and Communication Technology
        
        \vspace{1.5cm}
        
        \textbf{Distributed Systems}
        
        \vspace{2cm}
        
        \textbf{PRACTICAL WORK 1: TCP FILE TRANSFER}
        
        \vspace{1.5cm}
        
        \textbf{Report by:} \\
        \vspace{0.5cm}
        Name 1 - ID \\
        Name 2 - ID \\
        Name 3 - ID \\
        % Điền thêm tên thành viên nhóm vào đây
        
        \vfill
        
        \vspace{0.8cm}
        
        \Large
        Hanoi, \the\year
        
    \end{center}
\end{titlepage}

\tableofcontents
\newpage


\section{Thiết kế giao thức (Protocol Design)}
Giao thức truyền file được xây dựng dựa trên giao thức TCP/IP để đảm bảo độ tin cậy. Quy trình truyền tải bao gồm các bước sau:

\begin{enumerate}
    \item \textbf{Khởi tạo kết nối:} Client kết nối đến Server thông qua địa chỉ IP và Port đã định trước (Handshake).
    \item \textbf{Gửi metadata:} Client gửi tên file (được encode UTF-8) sang Server.
    \item \textbf{Xác nhận (Acknowledgement):} Server nhận tên file, chuẩn bị file đích để ghi và gửi tín hiệu "ACK" lại cho Client để báo hiệu sẵn sàng nhận dữ liệu.
    \item \textbf{Truyền dữ liệu:} Sau khi nhận được "ACK", Client mở file cục bộ, đọc từng khối dữ liệu (1024 bytes) và gửi liên tục sang Server.
    \item \textbf{Kết thúc:} Khi đọc hết file, Client đóng kết nối. Server nhận thấy không còn dữ liệu (EOF), tiến hành đóng file và đóng kết nối socket.
\end{enumerate}

Dưới đây là biểu đồ tuần tự (Sequence Diagram) mô tả giao thức:

\begin{figure}[H]
    \centering
    \begin{tikzpicture}[node distance=4cm, auto, >=stealth']
        % Nodes
        \node[] (client) {\textbf{Client}};
        \node[right=of client] (server) {\textbf{Server}};
        
        % Lines
        \draw[thick] (client) -- ++(0,-8);
        \draw[thick] (server) -- ++(0,-8);
        
        % Arrows
        % Connect
        \draw[->] ($(client)+(0,-1)$) -- node[midway, above] {Connect (SYN)} ($(server)+(0,-1)$);
        
        % Send Filename
        \draw[->] ($(client)+(0,-2)$) -- node[midway, above] {Send Filename (UTF-8)} ($(server)+(0,-2)$);
        
        % ACK
        \draw[<-] ($(client)+(0,-3)$) -- node[midway, above] {Send "ACK"} ($(server)+(0,-3)$);
        
        % Data Stream
        \draw[->, dashed] ($(client)+(0,-4.5)$) -- node[midway, above] {File Data (Chunk 1)} ($(server)+(0,-4.5)$);
        \draw[->, dashed] ($(client)+(0,-5.0)$) -- node[midway, above] {File Data (Chunk 2)} ($(server)+(0,-5.0)$);
        \draw[->, dashed] ($(client)+(0,-5.5)$) -- node[midway, above] {...} ($(server)+(0,-5.5)$);
        
        % Close
        \draw[->] ($(client)+(0,-7)$) -- node[midway, above] {Close Connection (EOF)} ($(server)+(0,-7)$);
        
    \end{tikzpicture}
    \caption{Biểu đồ giao thức truyền file qua TCP}
    \label{fig:protocol}
\end{figure}

\section{Tổ chức hệ thống (System Organization)}
Hệ thống được tổ chức theo mô hình Client-Server cơ bản[cite: 1, 1392].
\begin{itemize}
    \item \textbf{Server:} Chạy trên máy đích, lắng nghe tại cổng 65432. Server chịu trách nhiệm chấp nhận kết nối, nhận luồng byte và ghi xuống ổ đĩa cứng với tên file được thêm tiền tố "received\_".
    \item \textbf{Client:} Chạy trên máy nguồn, chịu trách nhiệm đọc file từ ổ đĩa và đẩy dữ liệu qua socket.
\end{itemize}

\begin{figure}[H]
    \centering
    \begin{tikzpicture}[
        block/.style={rectangle, draw, fill=blue!20, text width=5em, text centered, rounded corners, minimum height=4em},
        line/.style={draw, -latex'},
        cloud/.style={draw, ellipse,fill=red!20, node distance=3cm, minimum height=2em}
    ]
    
    % Nodes
    \node [block] (client) {Client App (client.py)};
    \node [cloud, right=of client, node distance=4cm] (network) {Internet/ Localhost};
    \node [block, right=of network, node distance=4cm] (server) {Server App (server.py)};
    
    \node [draw, cylinder, shape border rotate=90, aspect=0.25, fill=yellow!30, below=of client, minimum height=1.5cm] (fs_client) {File System (Input)};
    \node [draw, cylinder, shape border rotate=90, aspect=0.25, fill=yellow!30, below=of server, minimum height=1.5cm] (fs_server) {File System (Output)};

    % Lines
    \path [line] (fs_client) -- node [midway, left] {Read} (client);
    \path [line] (client) -- node [midway, above] {Send} (network);
    \path [line] (network) -- node [midway, above] {Recv} (server);
    \path [line] (server) -- node [midway, right] {Write} (fs_server);

    \end{tikzpicture}
    \caption{Kiến trúc hệ thống truyền file}
    \label{fig:architecture}
\end{figure}

\section{Cài đặt (Implementation)}
Hệ thống được cài đặt bằng ngôn ngữ Python sử dụng thư viện `socket` chuẩn.

\subsection{Phía Server (server.py)}
Server thực hiện bind địa chỉ, lắng nghe, gửi ACK và vòng lặp `while True` để nhận dữ liệu cho đến khi Client ngắt kết nối.

\lstinputlisting[language=Python, caption=Mã nguồn Server]{server.py}

\subsection{Phía Client (client.py)}
Client thực hiện kết nối, gửi tên file trước, đợi phản hồi ACK từ server để đảm bảo đồng bộ, sau đó mới gửi dữ liệu file.

\lstinputlisting[language=Python, caption=Mã nguồn Client]{client.py}

\section{Phân công công việc (Roles)}
Dưới đây là bảng phân công công việc của các thành viên trong nhóm:

\begin{table}[H]
    \centering
    \begin{tabular}{|c|l|l|}
        \hline
        \textbf{STT} & \textbf{Thành viên} & \textbf{Công việc} \\
        \hline
        1 & Nguyễn Văn A & Viết mã nguồn Server, thiết kế giao thức \\
        \hline
        2 & Trần Thị B & Viết mã nguồn Client, kiểm thử hệ thống \\
        \hline
        3 & Lê Văn C & Viết báo cáo, vẽ biểu đồ \\
        \hline
    \end{tabular}
    \caption{Bảng phân công công việc}
    \label{tab:roles}
\end{table}

\end{document}